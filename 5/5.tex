%%%%%%%%%%%%%%%%%%%%%%%%%%%%%%%%%%%%%%%%%%%%%%%
%%%This is a science homework template. Modify the preamble to suit your needs. 
%The junk text is   there for you to immediately see how the headers/footers look at first 
%typesetting.


\documentclass[12pt]{article}

%AMS-TeX packages
\usepackage{amssymb,amsmath,amsthm} 
%geometry (sets margin) and other useful packages
\usepackage[margin=1.25in]{geometry}
\usepackage{graphicx,ctable,booktabs}


%
%Redefining sections as problems
%
\makeatletter
\newenvironment{problem}{\@startsection
       {section}
       {1}
       {-.2em}
       {-3.5ex plus -1ex minus -.2ex}
       {2.3ex plus .2ex}
       {\pagebreak[3]%forces pagebreak when space is small; use \eject for better results
       \large\bf\noindent{P }
       }
       }
\makeatother

\makeatletter
\newenvironment{solution}{\@startsection
       {subsection}
       {2}
       {-.2em}
       {-3.5ex plus -1ex minus -.2ex}
       {2.3ex plus .2ex}
       {\pagebreak[3]%forces pagebreak when space is small; use \eject for better results
       \large\bf\noindent\emph{(sol) }
       }
       }
\makeatother

%
%Fancy-header package to modify header/page numbering 
%
\usepackage{fancyhdr}
\pagestyle{fancy}
%\addtolength{\headwidth}{\marginparsep} %these change header-rule width
%\addtolength{\headwidth}{\marginparwidth}
\lhead{Problem \thesection}
\chead{} 
\rhead{\thepage} 
\cfoot{} 
\rfoot{\footnotesize PS \#} 
\renewcommand{\headrulewidth}{.3pt} 
\renewcommand{\footrulewidth}{.3pt}
\setlength\voffset{-0.25in}
\setlength\textheight{648pt}

%%%%%%%%%%%%%%%%%%%%%%%%%%%%%%%%%%%%%%%%%%%%%%%

%
%Contents of problem set
%    
\begin{document}

\title{Title}
\author{Author}
\date{Date}

\maketitle

\thispagestyle{empty}

\begin{problem}{24}
    The process described in part (c) of Exercise 22 can of course be applied to functions that map $(0,\infty)$ to $(0,\infty)$.\\
    Fix some $\alpha>1$, and put
    $$f(x)=\frac{1}{2}(x+\frac{\alpha}{x}), \qquad g(x)=\frac{\alpha+x}{1+x}$$
    Both $f$ and $g$ have $\sqrt(\alpha)$ as their only fixed point in $(0,\infty)$. Try to explain, on the basis of properties of
    $f$ and $g$, why the convergence in Exercise 16, Chap. 3, is so much more rapid than it is in Exercise 17. (Compare $f'$ and
    $g'$, draw the zig-zags suggested in Exercise 22.) \\
    Do the same when $0<\alpha<1$.
\end{problem}
\begin{solution}{}
    Looking at the zig-zag, $f(x)$ does not spiral and stays on the positive side, and $g(x)$ goes back and forth, so presumably $f(x)$ converges
    quite a bit faster. For $\alpha < 1$, $g(x)$ no longer goes back and forth, but it's not clear if it converges as fast as $f(x)$. I want to say
    no because when $x$ gets close to the fixed point, $f$ will still be halved each time, but $g$ does not have a constant rate at which it converges. 
\end{solution}

\begin{problem}{25}
    Suppose $f$ is twice diffrentiable on $[a,b]$, $f(a)<0$, $f(b)>0$, $f'(x)\ge \delta > 0$, and $0 \le f"(x) \le M$ for all $x \in [a,b]$.
    Let $\xi$ be the unique point in $(a,b)$ at which $f(\xi)=0$.\\
    Complete the following outline of Newton's method for computing $\xi$.\\
    (a) Choose $x_1\in(\xi,b)$, and define $\{x_n\}$ by
    $$x_{n+1}=x_n -\frac{f(x_n)}{f'(x_n)}$$
    Interpret this geometrically, in terms of a tangent to the graph of $f$. \\
    (b) Prove that $x_{n+1} < x_n$ and that 
    $$\lim_{n\to\infty} x_n = \xi$$
    (c) Use Taylor's theorem to show that 
    $$x_{n+1}-\xi=\frac{f"(t_n)}{2f'(x_n)}(x_n-\xi)^2$$
    for some $t_n \in (\xi,x_n)$.\\
    (d) If $A= \frac{M}{2\delta}$, deduce that
    $$0 \le x_{n+1} - \xi \le \frac{1}{A}[A(x_1-\xi)]^{2^n}$$
    (e) Show that Newton's method amounts to finding a fixed point of the function $g$ defined by
    $$g(x)=x-\frac{f(x)}{f'(x)}$$
    How does $g'(x)$ behave for $x$ near $\xi$?\\
    (f) Apply to $f(x)=x^\frac{1}{3}$
\end{problem}
\begin{solution}{}
    (a) Geometrically this is taking the tangent line at $x_n$, finding the intersection with the x-axis, and applying that new x again.\\
    (b) Since both $f(x_n)$ and $f'(x_n)$ are positive, $x_{n+1}< x_n$. Since $\{x_n\}$ is monotonically decreasing, either it diverges to 
    $-\infty$ or converges to a limit point. Now for some $x_{n}$, we have by the mean value theorem
    $$\exists z \in (x_{n+1},x_n): f(x_n)-f(x_{n+1}) = f'(z)(x_n - x_{n+1})$$
    and by definition of $f$, $f'(x_n)(x_n-x_{n+1})=f(x_n)$, and since $f"$ is always positive, $f'(z) < f'(x_n)$, thus
    $$f(x_{n+1}) =f(x_n)- f'(z)(x_n - x_{n+1}) > f(x_n)-f'(x_n)(x_n - x_{n+1}) = f(x_n)-f(x_n) = 0$$
    so for all $x_n, f(x_n) > 0$, so $f(x_n)$ converges to 0, so $\{x_n\}$ converges to $\xi$. \\
    (c) Let $\beta = \xi$, $\alpha=x_n$, $n=2$. Then Taylor's theorem state that there exist a $t_n \in (\xi,x_n)$ such that
    $$f(\xi)=f(x_n)+f'(x_n)(\xi-x_n)+\frac{f"(t_n)}{2}(\xi-x_n)^2$$
    $$\implies -f(x_n)-f'(x_n)(\xi-x_n)=\frac{f"(t_n)}{2}(-(x_n-\xi))^2$$
    $$\implies -f(x_n)-f'(x_n)\xi+f'(x_n)x_n=\frac{f"(t_n)}{2}(x_n-\xi)^2$$
    divide by $f'(x_n)$
    $$\implies -\frac{f(x_n)}{f'(x_n)}-\xi+x_n=\frac{f"(t_n)}{2f'(x_n)}(x_n-\xi)^2$$
    $$\implies (x_n-\frac{f(x_n)}{f'(x_n)})-\xi=\frac{f"(t_n)}{2f'(x_n)}(x_n-\xi)^2$$
    $$\implies x_{n+1}-\xi=\frac{f"(t_n)}{2f'(x_n)}(x_n-\xi)^2$$
    (d) Will prove by induction. The base case of $n=1$ just follows from part c. \\
    The inductive step is as follows: using part c again
    $$\exists t_{n+1}: x_{n+2}-\xi = \frac{f"(t_{n+1})}{2f'(t_{n+1})}(x_{n+1}-\xi)^2$$
    $$\le A(x_{n+1}-\xi)^2 \le A(\frac{1}{A}(x_1-\xi)^{2^n})^2 = \frac{1}{A}(x_1-\xi)^{2^{(n+1)}}$$ 
    \qed
    (e) Follows directly from definition. $g'(x)$ goes to 1. \\
    (f) it keeps jumping back and forth and doesnt converge
\end{solution}

\begin{problem}{}
\end{problem}
\begin{solution}{}
\end{solution}

\end{document}
