%%%%%%%%%%%%%%%%%%%%%%%%%%%%%%%%%%%%%%%%%%%%%%%
%%%This is a science homework template. Modify the preamble to suit your needs. 
%The junk text is   there for you to immediately see how the headers/footers look at first 
%typesetting.


\documentclass[12pt]{article}

%AMS-TeX packages
\usepackage{amssymb,amsmath,amsthm} 
%geometry (sets margin) and other useful packages
\usepackage[margin=1.25in]{geometry}
\usepackage{graphicx,ctable,booktabs}


%
%Redefining sections as problems
%
\makeatletter
\newenvironment{problem}{\@startsection
       {section}
       {1}
       {-.2em}
       {-3.5ex plus -1ex minus -.2ex}
       {2.3ex plus .2ex}
       {\pagebreak[3]%forces pagebreak when space is small; use \eject for better results
       \large\bf\noindent{P }
       }
       }
\makeatother

\makeatletter
\newenvironment{solution}{\@startsection
       {subsection}
       {2}
       {-.2em}
       {-3.5ex plus -1ex minus -.2ex}
       {2.3ex plus .2ex}
       {\pagebreak[3]%forces pagebreak when space is small; use \eject for better results
       \large\bf\noindent\emph{(sol) }
       }
       }
\makeatother

%
%Fancy-header package to modify header/page numbering 
%
\usepackage{fancyhdr}
\pagestyle{fancy}
%\addtolength{\headwidth}{\marginparsep} %these change header-rule width
%\addtolength{\headwidth}{\marginparwidth}
\chead{} 
\rhead{\thepage} 
\cfoot{} 
\renewcommand{\headrulewidth}{.3pt} 
\renewcommand{\footrulewidth}{.3pt}
\setlength\voffset{-0.25in}
\setlength\textheight{648pt}

%%%%%%%%%%%%%%%%%%%%%%%%%%%%%%%%%%%%%%%%%%%%%%%

%
%Contents of problem set
%    
\begin{document}

\title{Title}
\author{Author}
\date{Date}

\maketitle

\thispagestyle{empty}

\begin{problem}{1}
    Suppose $\alpha$ increases on $[a,b]$, $a \le x_0 \le b$, $\alpha$ is continuous at $x_0$, $f(x_0)=1$, and
    $f(x)=0$ if $x \ne x_0$. Prove that $f$ is Riemann integrable and that $\int f \,d\alpha =0$.
\end{problem}
\begin{solution}{}
    Since $f$ is bounded and only has 1 discontinuity, and since $\alpha$ is continuous where $f$ is not, theorem
    6.10 applies and $f$ is Riemann integrable. \\
    Because $f$ is bounded, and $\alpha$ is continuous at $x_0$, $\Delta\alpha \to 0$ implies $\int f \to 0$.
    \qed
\end{solution}

\begin{problem}{2}
    Suppose $f \ge 0$, $f$ is continuous on $[a,b]$, and $\int_{a}^{b}f(x)\,dx=0$. Prove that $f(x)=0$
    for all $x\in [a,b]$. (Compare this to Exercise 1).
\end{problem}
\begin{solution}{}
    Since $f \ge 0$, let $f(x_0) = M > 0$. Since $f$ is continuous, by the intermediate value theorem, it assumes
    a point $f(x_1) = T < M$. Then there exists a partition such that $L(P,f) > 0$ which implies 
    $\int_{a}^{b}f(x)\,dx >0$ which is a contradiction.
    \qed
\end{solution}

\begin{problem}{3}
    Define three functions: $\beta_1,\beta_3,\beta_3$ as follows: $\beta_j=0$ if $x < 0$, $\beta_j=1$ if $x > 0$ for
    $j=1,2,3$; and $\beta_1(0)=0,\beta_2(0)=1,\beta_3(0)=\frac{1}{2}$. Let $f$ be a bounded function on $[-1,1]$. \\
    (a) Prove that $f \in \mathcal{R}(\beta_1)$ if and only if $f(0+)=f(0)$ and that then
    $$\int f \, d\beta_1 = f(0)$$
    (b) State and prove a similar result for $\beta_2$. \\
    (c) Prove that $f \in \mathcal{R}(\beta_3)$ if and only if $f$ is continuous at 0. \\
    (d) If $f$ is continuous at 0 prove that 
    $$\int f \,d\beta_1 = \int f \,d\beta_2 = \int f \,d\beta_3 =f(0)$$
\end{problem}
\begin{solution}{}
    (a) Assume $f(0+)=f(0)$. We only need to consider an open neighborhood of 0 since theorem 6.10 takes care of everywhere
    else. Fix $\epsilon > 0$ and let $\delta > 0$ such that $x < \delta \implies |f(x)-f(0)| < \epsilon$. Let $P$ be a
    partition such that $\Delta x_i < \delta$ for all $i$. We only need to consider $x \ge 0$. In particular only the 
    partition such that $x_i = 0$. Then 
    $$U(P,f,\beta_1) - L(P,f,\beta_1) = (M_i - m_i)\Delta \beta_1(x_i) = (M_i - m_i) < \epsilon$$
    And thus $f \in \mathcal{R}(\beta_1)$. \\
    Assume $f \in \mathcal{R}(\beta_1)$. If $f(0+) \ne f(0)$, then we see that it $\Delta \beta_1$ around a neighborhood
    of $0+$ is always 1, so the upper and lower sums never converge to each other. \\
    (b) same as a \\
    (c) same as a and b \\
    (d) follows from fact that if a limit exists, the left and right hand limits are equal.
\end{solution}

\begin{problem}{4}
    If $f(x)=0$ for all irrational $x$, $f(x)=1$ for all rational $x$, prove that $f$ is not integrable on $a[b]$ for
    any $a < b$.
\end{problem}
\begin{solution}{}
    The rationals and irrationals are dense in each other, thus the upper sum is always equal to 1 while the lower sum
    is always 0. 
    \qed
\end{solution}

\begin{problem}{5}
    Suppose $f$ is a bounded real function on $[a,b]$, and $f^2 \in \mathcal{R}$ on $[a,b]$. Does it follow that 
    $f\in \mathcal{R}$? Does the answer change if we assume that $f^3 \in \mathcal{R}$?
\end{problem}
\begin{solution}{}
    No. Take the last problems function but let $f(x)=1$ if x is rational and $-1$ if irrational. $f^2$ is the constant
    function 1 which is integrable, but $f$ is not. If $f^3$ is integrable then so is $f$ since the cube root is 
    continuous on the real line. 
    \qed
\end{solution}

\begin{problem}{6}
    Let $P$ be the Contor set constructed in Sec. 2.44. Let $f$ be a bounded real function on $[0,1]$ which is continuous
    at every point outside $P$. Prove that $f \in \mathcal{R}$ on $[0,1]$. \\
    HINT: $P$ can be covered by finitely many segments whose total length can be made as small as desired. Proceed as in
    Theorem 6.10.
\end{problem}
\begin{solution}{}
    Cantor set has measure 0. Rest follows as in 6.10.
\end{solution}

\begin{problem}{7}
    Suppose $f$ is a real function on $(0,1]$ and $f \in \mathcal{R}$ on $[c,1]$ for every $c>0$. Define
    $$\int_{0}^{1}f(x) \,dx = \lim_{c\to 0} \int_{c}^{1}f(x) \,dx$$
    if this limit exists (and is finite).\\
    (a) If $f \in \mathcal{R}$ on $[0,1]$, show that this definition of the integral agrees with the old one.\\
    (b) Construct a function $f$ such that the above limit exists, although it fails to exist with $|f|$ in place of $f$. 
\end{problem}
\begin{solution}{}
    Skip for now
\end{solution}

\begin{problem}{8}
    Suppose $f\in \mathcal{R}$ on $[a,b]$ for every $b > a$ where $a$ is fixed. Define
    $$\int_{a}^{\infty} f(x) \,dx = \lim_{b\to \infty} \int_{a}^{b} f(x) \,dx$$
    if this limit exists (and is finite). Int that case, we say that the integral on the left converges. If it also converges
    after $f$ has been replaced by $|f|$, it is said to converge absolutely. \\
    Assume that $f(x) \ge 0$ and that $f$ decreases monotonically on $[,\infty)$. Prove that 
    $$\int_{1}^{\infty} f(x)\,dx$$
    converges if and only if
    $$\sum_{n=1}^{\infty}f(n)$$
    converges. (This is the so-called "integral test" for convergence of series)
\end{problem}
\begin{solution}{}
    Assume $\sum_{n=1}^{\infty}f(n)$ converges. Then so does $\int_{1}^{\infty} f(x)\,dx$ since the partition $\{1,2,...,n\}$ has an
    Upper sum equal to the sum. Lower sums yield the other way. 
    \qed
\end{solution}

\begin{problem}{9}
    skip for now
\end{problem}
\begin{solution}{}  
\end{solution}

\begin{problem}{10}
    Let $p$ and $q$ be positive real numbers such that 
    $$\frac{1}{p} + \frac{1}{q} = 1$$
    Prove that following statements. \\
    (a) If $u \ge 0$ and $v \ge 0$, then
    $$uv \le \frac{u^p}{p} + \frac{v^q}{q}$$
    Equality holds if and only if $u^p = v^q$.\\
    (b) If $f \in \mathcal{R}(\alpha)$, $g \in \mathcal{R}(\alpha)$, $f\ge $, $g\ge 0$, and
    $$\int_{a}^{b}f^p \,d\alpha = 1 =  \int_{a}^{b}g^q \,d\alpha$$
    then
    $$\int_{a}^{b}fg \,d\alpha \le 1$$
    (c) If $f$ and $g$ are complex functions in $\mathcal{R}(\alpha)$, then
    $$\bigg |\int_{a}^{b} fg \,d\alpha \bigg| \le \bigg\{\int_{a}^{b} |f|^p \,d\alpha \bigg\}^{1/p} \bigg\{\int_{a}^{b} |g|^q \,d\alpha \bigg\}^{1/q}$$
    (d) Show that Holder's inequality is also true for the "improper" integrals described in exercise 7 and 8.
\end{problem}
\begin{solution}{}  
    (a) $$\ln(uv) = \ln(u) + \ln(v) = \frac{\ln(u)*p}{p} + \frac{\ln(v)*q}{q}$$
    $$=\frac{\ln(u^p)}{p} + \frac{\ln(v^q)}{q} \le \ln(\frac{u^p}{p}+\frac{v^q}{q})$$
    The last inequality due to concavity of $\ln$. \\
    (b) from part a let
    $$fg \le \frac{f^p}{p}+\frac{g^q}{q}$$
    then
    $$\int_{a}^{b} fg \le \int_{a}^{b}\frac{f^p}{p}+\int_{a}^{b}\frac{g^q}{q}$$
    $$=\frac{\int_{a}^{b}f^p}{p}+\frac{\int_{a}^{b}g^q}{q}$$
    $$=\frac{1}{p}+\frac{1}{q} = 1$$
    (c) replace $f$ and $g$ with 
    $\frac{|f|}{(\int_{a}^{b}|f|^p\,d\alpha)^{1/p}}$ and $\frac{|g|}{(\int_{a}^{b}|g|^q\,d\alpha)^{1/q}}$
    and integrate both sides. The right hand side becomes 1, and then multipling both sides by the left hand side's denominator,
    and the result follows.\\
    (d) skip for now.
\end{solution}

\begin{problem}{11}
    Let $\alpha$ be a fixed increasing function on $[a,b]$. For $u \in \mathcal{R}(\alpha)$, define
    $$||u||_2 = \bigg \{\int_{a}^{b}|u|^2\,d\alpha \bigg\}^{1/2}$$
    Suppose $f,g,h \in \mathcal{R}(\alpha)$ and prove the triangle inequality
    $$||f-h||^2 \le ||f-g||_2 + ||g-h||_2$$
\end{problem}
\begin{solution}{}
    $$||f-h||_2 = \bigg (\int_{a}^{b} |f-h|^2 \bigg )^{1/2}$$
    $$= \bigg (\int_{a}^{b} |(f-g)+(g-h)|^2 \bigg )^{1/2}$$
    $$= \bigg (\int_{a}^{b} |(f-g)|^2+2|f-g||g-h|+|(g-h)|^2 \bigg )^{1/2}$$
    $$= \bigg (\int_{a}^{b} |(f-g)|^2+2\int_{a}^{b}|f-g||g-h|+\int_{a}^{b}|(g-h)|^2 \bigg )^{1/2}$$
    $$\le \bigg (||(f-g)||_2+2||f-g||_2||g-h||_2+||(g-h)||_2 \bigg )^{1/2}$$
    $$= \bigg ((||(f-g)||+||(g-h)||)^2 \bigg )^{1/2}$$
    \qed
\end{solution}

\begin{problem}{12}
    With the notations of Exercise 11, suppose $f\in \mathcal{R}(\alpha)$ and $\epsilon>0$. Prove that there
    exists a continuous function $g$ on $[a,b]$ such that $||f-g||_2 < \epsilon$.\\
    Hint: Let $P=\{x_0,...,x_n\}$ be a suitable partition of $[a,b]$, define 
    $$g(t)=\frac{x_i-t}{\Delta x_i}f(x_{i-1})+\frac{t-x_{i-1}}{\Delta x_i}f(x_i)$$
    if $x_{i-1} \le t \le x_i$.
\end{problem}
\begin{solution}{}
    $g$ is continuous since it's a linear combination of $f$ which is integrable so with respect to some partition,
    $g$ is continuous. Then
    $$f(t)-g(t)=\frac{x_i-t}{\Delta x_i}(f(t)-f(x_{i-1}))+\frac{t-x_{i-1}}{\Delta x_i}(f(t)-f(x_i))$$
    Then
    $$\int_{a}^{b}|f-g|^2\,d\alpha = \sum_{i=0}^{N}\int_{x_i}^{x_{i+1}}|f-g|^2\,d\alpha$$
    since $f$ is integrable, we can choose a partition such that $M_i-m_i < 2M$ where $M$ is $sup(f)$,
    and $\Delta \alpha(x) < \frac{\epsilon^2}{4M^2}$
    $$\le \sum_{i=0}^{N}(M_i-m_i)^2\Delta \alpha$$
    $$\le 4M^2\sum_{i=0}^{N}\Delta \alpha$$
    $$\le 4M^2(\frac{\epsilon^2}{4M^2}) = \epsilon^2$$
    \qed
\end{solution}

\begin{problem}{13}
    Define 
    $$f(x)=\int_{x}^{x+1}sin(t^2)dt$$
    (a) Prove that $|f(x)| < \frac{1}{x}$ if $x>0$.\\
    Hint: put $t^2=u$ and integrate by parts to show that $f(x)$ is equal to
    $$\frac{\cos(x^2)}{2x} - \frac{cos[x+1]}{2(x+1)} - \int_{x^2}^{(x+1)^2} \frac{\cos u}{4u^{3/2}}$$
    (b) Prove that 
    $$2xf(x)=\cos(x^2)-\cos[(x+1)^2] + r(x)$$
    where $|r(x)| < c/x$ and $c$ is a constant.\\
    (c) Find the upper and lower limits of $xf(x)$ as $x\to \infty$.\\
    (d)
\end{problem}
\begin{solution}{}
    (a) Using $u$ substitution and integration by parts we can get the result as in the hint. Then since $cos$ attains
    its max of $1$ we let $\cos u =1$ and integrate the remaining integral and plug in the ends points which grants
    $$\frac{1+cos(x^2)}{2x}-  \frac{1+cos[(x+1)^2]}{2(x+1)}$$
    the second term can never be negative since so the maximum of the whole expression is when it is zero so
    $$\le \frac{1+cos(x^2)}{2x}$$
    similarly cosine's max is 1 so
    $$\le \frac{1+1}{2x} = \frac{1}{x}$$
    The fact that cosine's min is $-1$ yields that
    $$f(x) \ge -\frac{1}{x}$$
    (b) multiply the right hand side of $f(x)$ by $2x$ and do the same thing as in (a), and the last integral can again be 
    bounded. \\
    (c) 
\end{solution}

\begin{problem}{14}
    basically same problem as 13, but converges faster.
\end{problem}
\begin{solution}{}
\end{solution}

\begin{problem}{15}
    Suppose $f$ is a real, continuously diffrentiable function on $[a,b]$, $f(a)=f(b)=0$, and
    $$\int_{a}^{b}f^2(x)\,dx=1$$
    Prove that 
    $$\int_{a}^{b}xf(x)f'(x)\,dx=-\frac{1}{2}$$
    and that 
    $$\int_{a}^{b}[f'(x)]^2\,dx \cdot \int_{a}^{b}x^2f^2(x)\,dx > \frac{1}{4}$$
\end{problem}
\begin{solution}{}
    $$\int_{a}^{b}F(x)g(x) = F(b)G(b)-F(a)G(a)-\int_{a}^{b}f(x)G(x)$$
    Integrate by parts and let $F=xf(x),g=f'(x)$, then
    $$\int_{a}^{b}xf(x)f'(x)\,dx= -\int_{a}^{b} f(x)(xf'(x)+f(x))\,dx$$
    so we have
    $$\int_{a}^{b}xf(x)f'(x)\,dx=-\int_{a}^{b}xf'(x)f(x)\,dx-\int_{a}^{b}f^2(x)\,dx$$
    $$2\int_{a}^{b}xf(x)f'(x)\,dx=-1$$
    $$\int_{a}^{b}xf(x)f'(x)\,dx=-\frac{1}{2}$$
    The inequality follows from Holder's inequality as proved in problem 10.
    \qed
\end{solution}

\begin{problem}{16}
    For $1<s<\infty$, define
    $$\zeta(s)=\sum_{n=1}^{\infty}\frac{1}{n^s}$$
    (This is Riemann's zeta function, of great importance in the study of the distribution of prime
    numbers.) Prove that\\
    (a) $\zeta(s) = s\int_{1}^{\infty}\frac{[x]}{x^{s+1}}\,dx$ \\
    and that
    (b) $\zeta(s) = \frac{s}{s-1}-s\int_{1}^{\infty}\frac{x-[x]}{x^{s+1}}\,dx$\\
    where $[x]$ denotes the greatest integer $\ge x$.\\
    Prove that the integral in (b) converges for all $s > 0$. 
\end{problem}
\begin{solution}{}
    (a) the floor function makes $x$ constant between integer intervals, thus
    $$s\int_{1}^{\infty}\frac{[x]}{x^{s+1}}\,dx = s \sum_{n=1}^{\infty}n\cdot \int_{n}^{n+1}\frac{1}{x^{s+1}}\,dx$$
    $$=s\sum_{n=1}^{\infty}n\cdot \Big[-\frac{1}{s}\cdot\frac{1}{x^s}\Big]_{n}^{n+1}$$
    $$=s\frac{1}{s}\sum_{n=1}^{\infty}n\cdot (\frac{1}{n^s}-\frac{1}{(n+1)^s})$$
    the sum partially telescopes into the wanted result.\\
    (b) By linearity of integrals, we only need to show that
    $$\frac{s}{s-1} = s \cdot \int_{1}^{\infty} \frac{1}{x^s}$$
    which is clear. Convergence is also clear from b.
\end{solution}

\begin{problem}{17}
    Suppose $\alpha$ monotonically on $[a,b]$, $g$ is continuous, and $g(x)=G'(x)$ for $a \le x \le b$. Prove that
    $$\int_{a}^{b}\alpha(x)g(x)\,dx=G(b)\alpha(b)-G(a)\alpha(a)-\int_{a}^{b}G\,d\alpha$$
    Hint: Take $g$ real, without loss of generality.
\end{problem}
\begin{solution}{}  
\end{solution}

\begin{problem}{}
\end{problem}
\begin{solution}{}  
\end{solution}

\end{document}
