%%%%%%%%%%%%%%%%%%%%%%%%%%%%%%%%%%%%%%%%%%%%%%%
%%%This is a science homework template. Modify the preamble to suit your needs. 
%The junk text is   there for you to immediately see how the headers/footers look at first 
%typesetting.


\documentclass[12pt]{article}

%AMS-TeX packages
\usepackage{amssymb,amsmath,amsthm} 
%geometry (sets margin) and other useful packages
\usepackage[margin=1.25in]{geometry}
\usepackage{graphicx,ctable,booktabs}


%
%Redefining sections as problems
%
\makeatletter
\newenvironment{problem}{\@startsection
       {section}
       {1}
       {-.2em}
       {-3.5ex plus -1ex minus -.2ex}
       {2.3ex plus .2ex}
       {\pagebreak[3]%forces pagebreak when space is small; use \eject for better results
       \large\bf\noindent{P }
       }
       }
\makeatother

\makeatletter
\newenvironment{solution}{\@startsection
       {subsection}
       {2}
       {-.2em}
       {-3.5ex plus -1ex minus -.2ex}
       {2.3ex plus .2ex}
       {\pagebreak[3]%forces pagebreak when space is small; use \eject for better results
       \large\bf\noindent\emph{(sol) }
       }
       }
\makeatother

%
%Fancy-header package to modify header/page numbering 
%
\usepackage{fancyhdr}
\pagestyle{fancy}
%\addtolength{\headwidth}{\marginparsep} %these change header-rule width
%\addtolength{\headwidth}{\marginparwidth}
\chead{} 
\rhead{\thepage} 
\cfoot{} 
\renewcommand{\headrulewidth}{.3pt} 
\renewcommand{\footrulewidth}{.3pt}
\setlength\voffset{-0.25in}
\setlength\textheight{648pt}

%%%%%%%%%%%%%%%%%%%%%%%%%%%%%%%%%%%%%%%%%%%%%%%

%
%Contents of problem set
%    
\begin{document}

\title{Chapter 7}
\author{Campinghedgehog}
\date{June 14, 2023}

\maketitle

\thispagestyle{empty}

\begin{problem}{2}
\end{problem}
\begin{solution}{}
    Since both sequences converge uniformly (to $f$ and $g$), fix $\epsilon$ such that
    $$|f_n(x)-f(x)| < \frac{\epsilon}{2}$$ 
    for all $x$ when $n > N$ and 
    $$|g_m(x)-g(x)| < \frac{\epsilon}{2}$$
    for all $x$ when $m>M$. Let $T=max(N,M)$. Then if $n > T$,
    $$|f_n(x)+g_n(x) - f(x)-g(x)| = |f_n(x)-f(x)+g_n(x)-g(x)|$$
    $$\le  |f_n(x)-f(x)|+|g_n(x)-g(x)| = \epsilon$$
    Next assume that both sequences are of bounded functions. From problem 1, we can conclude that both
    sequences are uniformly bounded, that is there exists $T$ such that
    $$|f_i(x)| < T$$
    for all $x$ and $i$ and there exists $P$ such that
    $$|g_i(x)| < P$$
    for all $x$ and $i$. Then let $M=max(T,P)$. Then
    $$|f_n(x)g_n(x)-f(x)g(x)| = |f_n(x)g_n(x)-f(x)g_n(x)+f(x)g_n(x)-f(x)g(x)|$$
    $$=|g_n(x)(f_n(x)-f(x))+f(x)(g_n(x)-g(x))| \le |M(f_n(x)-f(x))+M(g_n(x)-g(x))|$$
    $$\le M|(f_n(x)-f(x))|+M|(g_n(x)-g(x))|$$
    and since the sequences are uniformly continuous, for all $\epsilon > 0$
    $$< M\epsilon+M\epsilon = 2M\epsilon$$
    thus $\{f_ng_n\}$ is uniformly convergent.
    \qed
\end{solution}

\begin{problem}{3}
    Construct sequences $\{f_n\}$, $\{g_n\}$ which converge uniformly on some set $E$, but such that $\{f_ng_n\}$ does 
    not converge uniformly on $E$.
\end{problem}
\begin{solution}{}
    $f_n(x)=x$ for all $x$ converges to $f(x)=x$ uniformly with $\epsilon=0$ for all $n$. 
    $g_n(x)=\frac{1}{n}$ for all $x$ is the same. So $f_n(x)g_n(x)=0$. But then for any $n$, there exists 
    an $x$ equal to it, thus for all $n$ there exists some $x$ such that $f_n(x)g_n(x)=\frac{x}{n}=1 > \epsilon$.
    \qed
\end{solution}

\begin{problem}{4}
    Consider
    $$f(x)=\sum_{n=1}^{\infty}\frac{1}{1+n^2x}$$
    For what values of $x$ does the series converge absolutely? On what intervals does it converge uniformly?
    On what intervals does it fail to converge uniformly? Is $f$ continuous wherever the series converges?
    Is $f$ bounded?
\end{problem}
\begin{solution}{}
    The series is almost convergent (comparing with $\frac{1}{n^2}$), except for a few points. When $x=0$ 
    we get $\sum_{n=1}^{\infty}1$ which diverges. There are countable number of points where a term is 
    undefined, namely $-\frac{1}{n^2}$ making the denominator 0.\\
    For any $x>0$ it converges uniformly since given some $\epsilon>0$, $\frac{1}{1+n^2x} \le \frac{1}{n^2\epsilon}$ and the right
    hand side converges uniformly. The same is true for $x<0$ except for the aformentioned countable points.\\
    Any interval containing 0 on the boundary does not converge uniformly. We can always pick an $x$ so that the partial sums never
    converge.\\
    $f$ does not converge at $x=0$ so it's not bounded.
\end{solution}

\begin{problem}{5}
    Let
    \begin{equation}
        f_n(x)=
        \begin{cases}
            0, & \bigg (x < \frac{1}{n+1} \bigg ) \\
            \sin^2\frac{\pi}{x}, & \bigg (\frac{1}{n+1} \le x \le \frac{1}{n} \bigg) \\
            0, &  \bigg(\frac{1}{n} < x \bigg) \\
        \end{cases}
    \end{equation}
    Show that $\{f_n\}$ converges to a continuous function, but not uniformly. Use the series $\sum f_n$ to
    show that the absolute convergence, even for all $x$, does not imply uniform convergence.
\end{problem}
\begin{solution}{}
\end{solution}

\begin{problem}{}
\end{problem}
\begin{solution}{}
\end{solution}

\end{document}
