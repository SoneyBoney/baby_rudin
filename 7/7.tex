%%%%%%%%%%%%%%%%%%%%%%%%%%%%%%%%%%%%%%%%%%%%%%%
%%%This is a science homework template. Modify the preamble to suit your needs. 
%The junk text is   there for you to immediately see how the headers/footers look at first 
%typesetting.


\documentclass[12pt]{article}

%AMS-TeX packages
\usepackage{amssymb,amsmath,amsthm} 
%geometry (sets margin) and other useful packages
\usepackage[margin=1.25in]{geometry}
\usepackage{graphicx,ctable,booktabs,mathrsfs}


%
%Redefining sections as problems
%
\makeatletter
\newenvironment{problem}{\@startsection
       {section}
       {1}
       {-.2em}
       {-3.5ex plus -1ex minus -.2ex}
       {2.3ex plus .2ex}
       {\pagebreak[3]%forces pagebreak when space is small; use \eject for better results
       \large\bf\noindent{P }
       }
       }
\makeatother

\makeatletter
\newenvironment{solution}{\@startsection
       {subsection}
       {2}
       {-.2em}
       {-3.5ex plus -1ex minus -.2ex}
       {2.3ex plus .2ex}
       {\pagebreak[3]%forces pagebreak when space is small; use \eject for better results
       \large\bf\noindent\emph{(sol) }
       }
       }
\makeatother

%
%Fancy-header package to modify header/page numbering 
%
\usepackage{fancyhdr}
\pagestyle{fancy}
%\addtolength{\headwidth}{\marginparsep} %these change header-rule width
%\addtolength{\headwidth}{\marginparwidth}
\chead{} 
\rhead{\thepage} 
\cfoot{} 
\renewcommand{\headrulewidth}{.3pt} 
\renewcommand{\footrulewidth}{.3pt}
\setlength\voffset{-0.25in}
\setlength\textheight{648pt}

%%%%%%%%%%%%%%%%%%%%%%%%%%%%%%%%%%%%%%%%%%%%%%%

%
%Contents of problem set
%    
\begin{document}

\title{Chapter 7}
\author{Campinghedgehog}
\date{June 14, 2023}

\maketitle

\thispagestyle{empty}

\begin{problem}{1}
    Prove that every bounded uniformly convergent sequence of bounded functions is uniformly bounded.
\end{problem}
\begin{solution}{}
    Pick $N$ such that if $n,m>=N$ 
    $$|f_n(x)-f_m(x)|<1$$
    for all $x$. Then let $M=max(M_1,...,M_N)$ where $|f_i| < M_i$. Then let $P=M+1$, then
    if $i<N$, $|f_i| < M$ and if $i>N$ then $|f_i| < |f_N| + 1 \le M+1$.
    \qed
\end{solution}

\begin{problem}{2}
\end{problem}
\begin{solution}{}
    Since both sequences converge uniformly (to $f$ and $g$), fix $\epsilon$ such that
    $$|f_n(x)-f(x)| < \frac{\epsilon}{2}$$ 
    for all $x$ when $n > N$ and 
    $$|g_m(x)-g(x)| < \frac{\epsilon}{2}$$
    for all $x$ when $m>M$. Let $T=max(N,M)$. Then if $n > T$,
    $$|f_n(x)+g_n(x) - f(x)-g(x)| = |f_n(x)-f(x)+g_n(x)-g(x)|$$
    $$\le  |f_n(x)-f(x)|+|g_n(x)-g(x)| = \epsilon$$
    Next assume that both sequences are of bounded functions. From problem 1, we can conclude that both
    sequences are uniformly bounded, that is there exists $T$ such that
    $$|f_i(x)| < T$$
    for all $x$ and $i$ and there exists $P$ such that
    $$|g_i(x)| < P$$
    for all $x$ and $i$. Then let $M=max(T,P)$. Then
    $$|f_n(x)g_n(x)-f(x)g(x)| = |f_n(x)g_n(x)-f(x)g_n(x)+f(x)g_n(x)-f(x)g(x)|$$
    $$=|g_n(x)(f_n(x)-f(x))+f(x)(g_n(x)-g(x))| \le |M(f_n(x)-f(x))+M(g_n(x)-g(x))|$$
    $$\le M|(f_n(x)-f(x))|+M|(g_n(x)-g(x))|$$
    and since the sequences are uniformly continuous, for all $\epsilon > 0$
    $$< M\epsilon+M\epsilon = 2M\epsilon$$
    thus $\{f_ng_n\}$ is uniformly convergent.
    \qed
\end{solution}

\begin{problem}{3}
    Construct sequences $\{f_n\}$, $\{g_n\}$ which converge uniformly on some set $E$, but such that $\{f_ng_n\}$ does 
    not converge uniformly on $E$.
\end{problem}
\begin{solution}{}
    $f_n(x)=x$ for all $x$ converges to $f(x)=x$ uniformly with $\epsilon=0$ for all $n$. 
    $g_n(x)=\frac{1}{n}$ for all $x$ is the same. So $f_n(x)g_n(x)=0$. But then for any $n$, there exists 
    an $x$ equal to it, thus for all $n$ there exists some $x$ such that $f_n(x)g_n(x)=\frac{x}{n}=1 > \epsilon$.
    \qed
\end{solution}

\begin{problem}{4}
    Consider
    $$f(x)=\sum_{n=1}^{\infty}\frac{1}{1+n^2x}$$
    For what values of $x$ does the series converge absolutely? On what intervals does it converge uniformly?
    On what intervals does it fail to converge uniformly? Is $f$ continuous wherever the series converges?
    Is $f$ bounded?
\end{problem}
\begin{solution}{}
    The series is almost convergent (comparing with $\frac{1}{n^2}$), except for a few points. When $x=0$ 
    we get $\sum_{n=1}^{\infty}1$ which diverges. There are countable number of points where a term is 
    undefined, namely $-\frac{1}{n^2}$ making the denominator 0.\\
    For any $x>0$ it converges uniformly since given some $\epsilon>0$, $\frac{1}{1+n^2x} \le \frac{1}{n^2\epsilon}$ and the right
    hand side converges uniformly. The same is true for $x<0$ except for the aformentioned countable points.\\
    Any interval containing 0 on the boundary does not converge uniformly. We can always pick an $x$ so that the partial sums never
    converge.\\
    $f$ does not converge at $x=0$ so it's not bounded.
\end{solution}

\begin{problem}{5}
    Let
    \begin{equation}
        f_n(x)=
        \begin{cases}
            0, & \bigg (x < \frac{1}{n+1} \bigg ) \\
            \sin^2\frac{\pi}{x}, & \bigg (\frac{1}{n+1} \le x \le \frac{1}{n} \bigg) \\
            0, &  \bigg(\frac{1}{n} < x \bigg) \\
        \end{cases}
    \end{equation}
    Show that $\{f_n\}$ converges to a continuous function, but not uniformly. Use the series $\sum f_n$ to
    show that the absolute convergence, even for all $x$, does not imply uniform convergence.
\end{problem}
\begin{solution}{}
    The function converges (point-wise at least) to $f=0$, since rationals are dense in reals, fixing any $x$
    there is always an $n$ such that $f_n(x)=0$.\\
    The series cannot converge uniformly because for any $n$, there exists some $m$ such that 
    $x=\frac{1}{2m+\frac{1}{2}}$ such that $\frac{1}{n+1} < x < \frac{1}{n+1}$, thus
    $f(x) = \sin^2(2\pi m+\frac{\pi}{2})=1$.
    \qed
\end{solution}

\begin{problem}{6}
    Prove that the series
    $$\sum_{n=1}^{\infty} (-1)^n \frac{x^2+n}{n^2}$$
    converges uniformly in every bounded interval, but does not converge absolutely for any value of $x$.
\end{problem}
\begin{solution}{}
    Skip for now.
\end{solution}

\begin{problem}{7}
    For $n=1,2,3,...$, $x$ real, put
    $$f_n(x)=\frac{x}{1+nx^2}$$
    Show that $\{f_n\}$ converges uniformly to a function $f$, and that the equation
    $$f'(x)=\lim_{n \to \infty} f'_n(x)$$
    is correct if $x \ne 0$, but false if $x=0$.
\end{problem}
\begin{solution}{}
    Using the quotient rule and differentiating $f_n$ and setting it equal to $0$ shows that $f_n$ has
    its max at $x=\frac{1}{\sqrt{n}}$, at which $f_n(x)=\frac{1}{2\sqrt{n}}$. Thus the sequence $f_n$
    is bounded by $\frac{1}{2\sqrt{n}}$ which converges uniformly to $f=0$. \\
    Since 
    $$f'_n(x)=\frac{1-nx^2}{(1+nx^2)^2}$$
    so
    $$f'_n(0)=1$$
    so $\lim_{n\to \infty}f'_n(x)=1$ but since $f=0$, $f'=0$.
    \qed
\end{solution}

\begin{problem}{8}
    If 
    \begin{equation}
        I(x)=
        \begin{cases}
            0, & \big (x \le 0 \big) \\
            1, &  \big(x > 0 \big) \\
        \end{cases}
    \end{equation}
    if $\{x_n\}$ is a sequence of distinct points of $(a,b)$, and if $\sum |c_n|$ converges, prove that 
    the series
    $$f(x)=\sum_{n=1}^{\infty} c_n I(x-x_n) \qquad (a \le x \le b)$$
    converges uniformly, and that $f$ is continuous for every $x \ne x_n$.
\end{problem}
\begin{solution}{}
    Since $I$ is just an indicator function, it follows that
    $$f_N(x) = \sum_{n=1}^{N} c_n I(x-x_n) \le \sum_{n=1}^{N} |c_n|$$. Since $\sum |c_n|$ converges absolutely,
    it follows that $f_n$ converges uniformly. \\
    $f$ will be continuous if each $f_n$ is. And each $f_n$ is continuous if $x \ne x_n$, since if it 
    "jumps" via the indicator function and misses or includes an extra term greater than some $\epsilon$ 
    in the summation if so.
\end{solution}

\begin{problem}{9}
    Let $\{f_n\}$ be a sequence of continuous functions which converges uniformly to a function $f$ on a 
    set $E$. Prove that 
    $$\lim_{n\to\infty}f_n(x_n)=f(x)$$
    for every sequence of points $x_n\in E$ such that $x_n \to x$, and $x \in E$. Is the converse of this true?
\end{problem}
\begin{solution}{}
    Decouple $x_n$ from $f_n$ and call it $x_k$. Then we have
    $$\lim_{k\to\infty}\lim_{n\to\infty}f_n(x_k) = \lim_{k\to\infty}f(x_k)$$
    by uniform convergence. Then 
    $$\lim_{k\to\infty}f(x_k) = f(x)$$. Or more rigorously,
    Fix $\epsilon$. Pick $N$ such that if $n>N$, $|f_n(x)-f(x)| < \epsilon$ for all $x$ (by uniform convergence).
    Then pick $M$ such that if $m>M$, $|x_m - x| < \delta$ such that 
    $|f(x_m)-f(x)| < \epsilon$ (since $f$ is continuous). 
    Then pick $k>max(N,M)$, thus
    $$|f_k(x_k) - f(x)| = |f_k(x_k) - f(x_k) + f(x_k) - f(x)|$$
    $$\le |f_k(x_k) - f(x_k)| + |f(x_k) - f(x)| < 2 \epsilon$$
    Probem 5 is a counter example.
\end{solution}

\begin{problem}{10}
    Letting $(x)$ denote that fractional part of the real number $x$, consider the function
    $$f(x)=\sum_{n=1}^{\infty}\frac{(nx)}{n^2} \qquad (x \text{ real})$$
    Find all discontinuities of $f$, and show that they form a countable dense set.
    Show that $f$ is neverthless Riemann-integrable on every bounded interval.
\end{problem}
\begin{solution}{}
    Discontinuities of $f$ would include all discontinuities of the partial sums
    $$f_N(x)=\sum_{n=1}^{N}\frac{(nx)}{n^2}$$
    The discontinuities are when $(nx)$ "jumps" back to $0$. This happens in all rationals since there will
    always be an $n$ equal to the denominator of $x$, thus $f$ is discontinuities at all rationals. And the 
    rationals are countable and dense in the reals.\\
    $f$ is Riemann-integrable if each $f_n$ in the sequence is also Rieman-integrable. There are a finite
    number of discontinuities given an $f_n$ since it is only a sum of $N$ terms. Thus $f$ is also
    Riemann-integrable.
    \qed
\end{solution}

\begin{problem}{11}
    Suppose $\{f_n\}$, $\{g_n\}$ are defined on $E$, and \\
    (a) $\sum f_n$ has uniformly bounded partial sums;\\
    (b) $g_n \to 0$ uniformly on $E$;\\
    (c) $g_1(x)\ge g_2(x)\ge g_3(x) \ge ...$ for every $x\in E$ \\
    Prove that $\sum f_ng_n$ converges uniformly on $E$. Hint: Compare with Theorem 3.42.
\end{problem}
\begin{solution}{}
    Choose $M$ such that $F_N = \sum_{n=1}^{N}f_n(x) < M$ for all $x$. Fix $\epsilon$. From b), choose $N$ such that 
    $g_N<\frac{\epsilon}{2M}$. Then for $p,q \ge N$, 
    $$|\sum_{n=p}^{q}f_ng_n|=|\sum_{n=p}^{q-1}F_n(g_n-g_{n+1})+F_qg_q-F_{p-1}g_p|$$
    $$\le M|\sum_{n=p}^{q-1}(g_n-g_{n+1})+g_q+b_p|$$
    $$=2g_pM\le 2g_NM\le \epsilon$$
    for all $x$.
    \qed
\end{solution}

\begin{problem}{12}
    Suppose $g$ and $f_n(n=1,2,3,...)$ are defined on $(0,\infty)$, are Riemann-integrable on 
    $[t,T]$ whenever $0<t<T<\infty$, $|f_n|\le g$, $f_n \to f$ uniformly on every compact subset of $(0,\infty)$
    and
    $$\int_{0}^{\infty} g(x) \,dx < \infty$$
    Prove that 
    $$\lim_{n\to \infty}\int_{0}^{\infty}f_n(x)\,dx=\int_{0}^{\infty}f(x)\,dx$$
\end{problem}
\begin{solution}{}
    Want to first show that $\int_{0}^{\infty}f_n(x)\,dx$ exists for all $n$. For $a<b$, 
    $$|\int_{a}^{t}f_n(x)\,dx \int_{b}^{t}f_n(x)\,dx| = |\int_{b}^{a}f_n(x)\,dx|$$
    $$\le \int_{b}^{a}|f_n(x)|\,dx \le \int_{b}^{a}g(x)\,dx$$
    And since $int_{0}^{\infty}g_n(x)\,dx$ exists, we can choose $a,b$ such that
    $int_{b}^{a}g(x)\,dx < \epsilon$ for all $\epsilon$. Similarly for the lower end point.
    It follows from Cauchy criterion that $\int_{0}^{\infty}f_n(x)\,dx$ exists.\\
    Because of uniform convergence, $f\le g$ and so $\int_{0}^{\infty}f(x)\,dx$ also exists. 
    Then
    $$\bigg|\int_{0}^{\infty}f_n(x)\,dx-\int_{0}^{\infty}f(x)\,dx\bigg|$$
    $$=\bigg|\int_{0}^{\infty}f_n(x)\,dx-\int_{a}^{b}f_n(x)\,dx
    +\int_{a}^{b}f_n(x)\,dx-\int_{a}^{b}f(x)\,dx
    +\int_{a}^{b}f(x)\,dx-\int_{0}^{\infty}f(x)\,dx\bigg|$$
    $$\le\bigg|\int_{0}^{\infty}f_n(x)\,dx-\int_{a}^{b}f_n(x)\,dx\bigg|$$
    $$+\bigg|\int_{a}^{b}f_n(x)\,dx-\int_{a}^{b}f(x)\,dx\bigg|
    +\bigg|\int_{a}^{b}f(x)\,dx-\int_{0}^{\infty}f(x)\,dx\bigg|$$
    $$=\bigg|\int_{0}^{\infty}f_n(x)\,dx-\int_{a}^{b}f_n(x)\,dx\bigg|$$
    $$+\bigg|\int_{a}^{b}f_n(x)-f(x)\,dx\bigg|
    +\bigg|\int_{a}^{b}f(x)\,dx-\int_{0}^{\infty}f(x)\,dx\bigg|$$
    With uniform convergence the middle term can be made arbitrarily small. The fact that the improper 
    integrals exist means that first and last terms can also be made arbitrarily small.
    \qed
\end{solution}

\begin{problem}{13}
    Assume that $\{f_n\}$ is a sequence of monotonically increasing functions on $\mathbb{R}^1$ with
    $0\le f_n(x) \le 1$ for all $x$ and all $n$.\\
    (a) Prove that there is a function $f$ and a sequence $\{n_k\}$ such that
    $$f(x)=\lim_{k\to \infty}f_{n_k}(x)$$
    for every $x\in \mathbb{R}^1$. (The existence of such a pointwise convergent subsequence is usually
    called Helly's selection theorem).\\
    (b) If, moreover, $f$ is continuous, prove that $f_{n_k}\to f$ uniformly on compact sets.
    Hint: (i) Some subsequence $\{f_{n_i}\}$ converges at all rational points $r$, say, to $f(r)$.
    (ii) Define $f(x)$, for any $x\in \mathbb{R}^1$, to be $\sup f(r)$, the $\sup$ taken over all $r\le x$.
    (iii) Show that $f_{n_i}(x) \to f(x)$ at every $x$ at which $f$ is continuous. (This is where monotonicity
    is strongly used.)
    (iv) A subsequence of $\{f_{n_i}\}$ converges at every point of discontinuity of $f$ since there are at most
    countably many such points. This proves (a). To prove (b), modify your proof of (iii) appropriately.
\end{problem}
\begin{solution}{}
    From Theorem 7.23 there exists a sequence of functions that converges on any countable set. Let the 
    countable set be the rationals. Then define $f(x)=\sup f(r)$ for $r\le x$. Points of continuity converge
    since the rationals are dense in the reals. For points of discontinuity, since the functions are monotonically
    increasing, there may only be a countable number of them; applying theorem 7.23 again to those points proves
    a). \\
    Skip b for now.
\end{solution}

\begin{problem}{14}
    Skip for now
\end{problem}
\begin{solution}{}
\end{solution}

\begin{problem}{15}
    Suppose $f$ is a continuous function on $\mathbb{R}^1$, $f_n(t)=f(nt)$ for $n=1,2,3,...$ and 
    $\{f_n\}$ is equicontinuous on $[0,1]$. What conclusion can you draw about $f$?
\end{problem}
\begin{solution}{}
    It has to be a constant function since if it is not, then using equicontinuity we can find 
    $x,y$ with $|f_n(x)-f_n(y)| > \epsilon$. Then $|f_n(\frac{x}{n})-f_n(\frac{y}{n})| > \epsilon$
    for all $n$, which contradicts equicontinuity.
    \qed 
\end{solution}

\begin{problem}{16}
    Suppose $\{f_n\}$ is an equicontinuous sequence of functions on a compact set $K$, and 
    $\{f_n\}$ converges pointwise on $K$. Prove that $\{f_n\}$ converges uniformly on $K$.
\end{problem}
\begin{solution}{}
    Fix $\epsilon$. Choose $N$ such that if $n>N$, then $|f_n(x)-f_n(y)| < \epsilon$ for all $x,y$
    if $|x-y| < \delta$. Then also choose a $\delta$-cover of $K$. That is there exists a finite number
    of open neighborhoods with center points $x_1,...,x_k$, with radius $\delta$. 
    Fixing each of the points, using point-wise convergence, choose $M$ such that if 
    $n,m > M$, $|f_n(x_i)-f_m(x_i)| < \epsilon$ for all $i$, which is finite.
    Then, for $n,m > max(N,M)$,
    $$|f_n(x)-f_m(x)|\le |f_n(x)-f_n(x_i)| + |f_n(x_i)-f_m(x_i)| + |f_m(x_i)-f_m(x)|$$
    Equicontinuity implies the first and third terms being less than $\epsilon$. The middle term is
    also less than $\epsilon$ since $n,m>M$. Thus
    $$|f_n(x)-f_m(x)| < 3\epsilon$$
    for all $x$.
    \qed
\end{solution}

\begin{problem}{17}
    Skip for now
\end{problem}
\begin{solution}{}
\end{solution}

\begin{problem}{18}
    Let $\{f_n\}$ be a uniformly bounded sequence of functions which are Riemann-integrable on $[a,b]$,
    and put
    $$F_n(x)=\int_{a}^{x}f_n(t)\,dt$$
    Prove that there exists a subsequence $\{F_{n_k}\}$ which converges uniformly on $[a,b]$.
\end{problem}
\begin{solution}{}
    Looks like it's probably going to use theorem 7.25. The domain is compact. We need to show that it is
    pointwise bounded and equicontinuous to use theorem 7.25. \\
    Pointwise boundedness follows since for all $n$,
    $$F_n(x)=\int_{a}^{x}f_n(t)\,dt \le M(b-a)$$
    $M$ comes from uniform boundedness. This applies for all $x$.\\
    Next, fix $\epsilon$. Then for some $\delta$, if $|x-y| < \delta$,
    $$|F_n(x)-F_n(y)| = |\int_{a}^{x}f_n(t)\,dt - \int_{a}^{y}f_n(t)\,dt|$$
    $$=|\int_{x}^{y}f_n(t)\,dt| \le \int_{x}^{y}|f_n(t)|\,dt$$
    $$\le M|x-y| < M\delta$$
    So if $\delta=\frac{\epsilon}{M}$, equicontinuity follows.
    \qed
\end{solution}

\begin{problem}{19}
    Let $K$ be a compact metric space, let $S$ be a subset of $\mathcal{C}(K)$. Prove that $S$ is compact
    (with respect to the metric defined in Section 7.14) if and only if $S$ is uniformly closed, pointwise
    bounded, and equicontinuous. (If $S$ is not equicontinuous, then $S$ contains a sequence which has 
    no equicontinuous subsequence, hence has no subsequence that converges uniformly on $K$).
\end{problem}
\begin{solution}{}
    Assume $S$ is compact. Then by Bolzano-Weirstrass, it must be closed and bounded. Assume to the contrary
    that $S$ is not equicontinuous. That is there exists some $\epsilon$ such that for all $\delta$,
    there exists some $x,y \in K$ and $f\in S$ such that
    $$|x-y| < \delta \quad \land \quad |f(x)-f(y)| \ge \epsilon$$
    Then there is a sequence $\{f_n\}$ such that there exists $x,y$ such that
    $$|f_n(x)-f_n(y) \ge \epsilon|$$
    for all $n$. Thus for any $n$, i.e. any subsequence, it cannot be uniformly convergent, so $S$ cannot 
    be compact.\\
    Assume $S$ is uniformly closed, pointwise bounded, and equicontinuous. By theorem 7.25, every sequence
    has a uniformly convergent subsequence, and so converges in the supremum norm, and thus $S$ is compact.
    \qed
\end{solution}

\begin{problem}{20}
    If $f$ is continuous on $[0,1]$ and if
    $$\int_{0}^{1}f(x)x^n\,dx=0 \qquad (n=0,1,2,...)$$
    prove that $f(x)=0$ on $[0,1]$. Hint: The integral of the product of $f$ with any polynomial is zero.
    Use the Weierstrass theorem to show that $\int_{0}^{1}f^2(x)\,dx=0$.
\end{problem}
\begin{solution}{}
    From the Weierstrass theorem, there exists a sequence of polynomials $\{p_n\}$ such that
    $p_n \to f$ uniformly. Thus
    $$\int_{0}^{1}f(x)(\lim_{n\to \infty}p_n(x))\,dx$$
    $$\lim_{n\to \infty}\int_{0}^{1}f(x)p_n(x)\,dx$$
    And from the hypothesis, apply linearity of the integral, and each term will be $0$.
    Thus,
    $$\int_{0}^{1}f^2(x)\,dx$$
    so $f^2=0$ so $f=0$.
    \qed
\end{solution}

\begin{problem}{21}
    Let $K$ be the unit circle in the complex plane (i.e., the set of all $z$ with $|z|=1$), and
    let $\mathcal{A}$ be the algebra of all functions of the form
    $$f(e^{i\theta})=\sum_{n=0}^{N}c_n e^{i n\theta} \qquad (\theta \text{ real})$$
    Then $\mathcal{A}$ separates points on $K$ and $\mathcal{A}$ vanishes at no point of $K$, but nevertheless
    there are continuous functions on $K$ which are not in the uniform closure of $\mathcal{A}$.\\
    Hint: For every $f\in \mathcal{A}$
    $$\int_{0}^{2\pi}f(e^{i\theta})e^{i\theta}\, d\theta=0$$
    and this is also true for every $f$ in the closure of $\mathscr{A}$.
\end{problem}
\begin{solution}{}
    The function $f(x) = \frac{1}{x}$ is continuous but
    $$\int_{0}^{2\pi}\frac{1}{e^{i\theta}}e^{i\theta}\, d\theta$$
    $$=\int_{0}^{2\pi}1\, d\theta=2\pi$$
    \qed
\end{solution}

\begin{problem}{22}
    Assume $f\in \mathscr{R}(\alpha)$ on $[a,b]$, and prove that there are polynomials $P_n$ such that 
    $$\lim_{n\to\infty}\int_{a}^{b}|f-P_n|^2\, d\alpha=0$$
    (Compare with Exercise 12, Chap. 6)
\end{problem}
\begin{solution}{}
    Follows from Weierstrass's theorem and Exercise 12 chapter 6 since the sequence of polynomials converges
    uniformly.
    \qed
\end{solution}

\begin{problem}{23}
    Skip for now.
\end{problem}
\begin{solution}{}
\end{solution}

\begin{problem}{24}
    Let $X$ be a metric space, with metric $d$. Fix a point $a\in X$. Assign to each $p\in X$ the function
    $f_p$ defined by
    $$f_p(x)=d(x,p)-d(x,a) \qquad (x\in X)$$
    Prove that $|f_p(x)\le d(a,p)$ for all $x\in X$, and that therefore $f_p\in \mathscr{C}(X)$.\\
    Prove that 
    $$||f_p-f_q||=d(p,q)$$
    for all $p,q \in X$.\\
    If $\phi(p)=f_p$ it follows that $\phi$ is an isometry (a distance-preserving mapping)
    of $X$ onto $\phi(X) \subset \mathscr{X}$.\\
    Let $Y$ be the closure of $\phi(X)$ in $\mathscr{X}$. Show that $Y$ is complete.\\
    $\emph{Conclusion: X is isometric to a dense subset of a complete metric space Y}$.\\
    (Exercise 24, Chap. 3 contains a different proof of this.)
\end{problem}
\begin{solution}{}
    
\end{solution}

\begin{problem}{}
\end{problem}
\begin{solution}{}
\end{solution}

\end{document}
